\documentclass[12pt, conference, compsocconf]{../IEEEtran}
\usepackage{xltxtra}
\usepackage{subfig}
\usepackage{booktabs}
\usepackage{flushend}
\usepackage[numbers,sort&compress]{natbib}
\setmainfont{Times New Roman}

\begin{document}

\title{Graph Reachability Queries: A Mini State-of-the-Art Survey}
\author
{
\IEEEauthorblockN
{
Hongjian Li
\IEEEauthorblockA
{
Department of Computer Science and Engineering\\
Chinese University of Hong Kong\\
hiji@cse.cuhk.edu.hk
}
}
}
\maketitle

\begin{abstract}

Background
Main focus
Discussion
Future

\end{abstract}

%\begin{IEEEkeywords}

%Reachability query, regular expression, graph segmentation, compression, label constraint, probabilistic query, tree decomposition, neural network

%\end{IEEEkeywords}

\section{Introduction}

Where the problem comes from?
Why it is important?

\section{Problem Definition}



\section{Existing Methods}

Comparison
Have your own evaluation
Use others results and give reasons

\citep{1063} is state-of-the-art review in 2010.

\citep{1052} Edges bear different types, indicating a variety of relationships. We propose a class of reachability queries and a class of graph patterns, in which an edge is specified with a regular expression of a certain form, expressing the connectivity in a data graph via edges of various types

\citep{1053} propose a hierarchical index based on graph segmentation to reduce index size without sacrificing query efficiency. We present experimental evidence to show that our approach can achieve significant space savings, and improve efficiency. We also show that our index need not be rebuilt for a large class of updates, a feature missing in all other contemporary systems.

\citep{1054} We propose a simple alternative based on a novel form of bit-vector compression. Our starting point is the observation that when computing the transitive closure, reachable vertices tend to cluster together. We adapt the well-known scheme of word-aligned hybrid compression (WAH) to work more efficiently by introducing word partitions. 

\citep{1055} given a label set S and two vertices u1 and u2 in a large directed graph G, we verify whether there exists a path from u1 to u2 under label constraint S. Several techniques are proposed in this paper to minimize the search space in computing path-label transitive closure.

\citep{1057} introduce a novel tree-based index framework which utilizes the directed maximal weighted spanning tree algorithm and sampling techniques to maximally compress the generalized transitive closure for the labeled graphs.

\citep{1056} each edge in a graph could be associated with a probability to appear. to determine if a source vertex could reach a destination vertex with probabilty larger than a user specified probability value t.

\citep{1058} present a novel indexing method based on the concept of tree decomposition. 

\citep{1059} GRAIL, that is based on the idea of randomized interval labeling. GRAIL is the only index that can scale to millions of nodes and edges.

\citep{1060} propose a new indexing scheme, called Path-Hop, which is even more space-efficient than those schemes based on 2-Hop cover and yet has query processing speed comparable to those chain/tree covers. 

\citep{1061} propose a hierarchical prediction framework, based on neural networks and a set of graph features and a knowledge base on past predictions, to select the optimal index for a graph database.

\section{Discussion}

Summary on current works
  Avoid overlapping with existing reviews

\section{Prospectives}

What will be done in the future? (Further emphasis on large graph)
  Consequences of current breakthroughs
  Bottleneck in the whole picture

\bibliographystyle{unsrtnat}
\bibliography{../refworks}

\end{document}
