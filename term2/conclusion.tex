\chapter{Conclusion}

Drug discovery is an expensive and long-term business. It takes about US\$1.8 billion over 13.5 years to develop a new drug \citep{716}. Hence computer-aided drug discovery is not only timesaving, but economics tells us this is the way we should be going. We aim to develop a comprehensive computational framework for structure-based drug discovery, simulating the early phases of modern drug discovery pipeline.

So far we have developed idock for fast predictions of both binding conformations of small compounds against given proteins and their binding affinities. idock can shortlist a few promising compounds out of millions for further clinical investigations. It is capable of screening 1.3 drug-like ligands per CPU minute on average, making it a competitive tool. Compared with Vina, idock achieved a speedup of 2.5x to 4.8x in terms of CPU time and a speedup of 6.3x to 10.4x in terms of elapsed time.

Moreover, on one hand, we are building istar as a SaaS platform for idock in order to automate the protein-ligand docking process. On the other hand, we are working on igrow for computational ligand synthesis in order to produce novel potent compounds. Overall speaking, we are gaining good progress.

\chapterend