\chapter{Conclusion}

\section{Conclusion}

Drug discovery is an expensive and long-term business. It takes about US\$1.8 billion over 13.5 years to develop a new drug \citep{716}. Drug discovery via merely biological and chemical means are both cost-inefficient and time-inefficient. Therefore we aim to develop a computational framework for structure-based drug discovery with GPU acceleration, simulating the early phases of modern drug discovery process in order to save money and time.

So far, we have developed two tools for this simulation purpose. The first tool, idock, is used for fast predictions of both binding conformations of small compounds against given proteins and their binding affinities. idock can shortlist a few promising compounds out of millions for further clinical investigations. It is capable of screening 1.3 drug-like ligands per CPU minute on average, making it a competitive tool. Compared with Vina, idock achieved a speedup of 2.5x to 4.8x in terms of CPU time and a speedup of 6.3x to 10.4x in terms of elapsed time. We have also utilized idock to identify compounds for potential replacement of TDF, trying to retaining the efficacy and meanwhile minimizing the toxicity.

The second tool, igrow, is used for computational synthesis of potent ligands. igrow helps to explore a much larger chemical space for novel drugs. It inherits the mutation and crossover operators from AutoGrow, and invents two new genetic operators, namely split and merging, significantly enriching ligand diversity. The split operator ensures that ligands will not grow excessively large. The merging operator is basically a reversed operator of split and aims to accelerate ligand growing. igrow implements Lipinski's \textit{Rule of Five} \citep{168} to ensure drug likeness. The program design is so flexible that it reserves room for adaptation to new chemical constraints. Its robust parser correctly processes two-letter chemical elements, and meanwhile adds additional support for phosphorus. igrow displays comparable performance in terms of predicted free energy but outperforms AutoGrow by 30\% in terms of execution time. Ligands generated by igrow never exceed 500 Da so that they can be absorbed by human body. 

We have used our three new tools together with some other existing tools to discover potential new drugs for the treatments of acquired immune deficiency syndrome (AIDS) and Alzheimer's disease (AD). Several promising ligands have been computationally identified and synthesized for further clinical investigations.

\chapterend
