Drug discovery is an expensive and long-term business. It takes US\$1.8 billion over 13.5 years to develop a new drug. Complementing laboratory experiments with computer simulations is obviously the right way to go. We therefore aim to develop a novel and concise computational framework for modern drug discovery.

We have developed idock for protein-ligand docking, trying to find inhibitors (i.e. ligands) of viral proteins of therapeutic interest. idock features built-in support for massive docking, a novel thread pool for parallel execution, automatic detection of inactive torsions, support for gzip/bzip2 compression, automatic docking recovery, detection of putative hydrogen bonds, per-atom binding affinity output, full reproducibility with code and data, as well as many other great features. Compared with state-of-the-art AutoDock Vina, idock features an amazing speedup of at least 8.69x and at most 37.51x, remarkably reducing the docking time from years to months given millions of ligands to dock.

We have developed istar as a SaaS (Software as a Service) platform for running programs online without software installation. We have ported idock and igrep onto istar. Compared with other online docking platforms such as DOCK Blaster and iScreen, istar features ligand filtering and previewing, progress monitoring in real time, innovative two-phase docking, slice-level parallelism, supplier list output, GPU acceleration support, full reproducibility with code and graphical tutorials, as well as many other great features. istar is currently supported by one Linux workstation and four Mac workstations in the background and available at http://istar.cse.cuhk.edu.hk.

We propose idock 2.0 to utilize GPU (Graphics Processing Unit) acceleration with both CUDA and OpenCL, harnessing the tremendous computational power and memory bandwidth offered by modern GPU nowadays. In the past we developed igrep, a CUDA implementation of the agrep algorithm, and thus have possessed expertise in GPU programming. We have surveyed the latest GPU architectures codenamed GK104 and Tahiti offered by NVIDIA and AMD respectively. We have also observed thread behaviors and identified program hotspots of idock 1.x. idock 2.0 will revolve around three strategies, which are maximizing parallel execution, maximizing memory bandwidth, and maximizing instruction throughput, aiming to further reduce the docking time from months to weeks and even to days, making large-scale docking a really pragmatic practice for potential use by computational chemists. By then we shall be able to perform proteomic-scale docking for the entire solved proteins and warehouse the results into a public database.

We propose idock 3.0 to enable computational synthesis of potent ligands. In the past we developed SmartGrow, trying to ``grow'' a novel ligand by merging two small ligands, or by exchanging parts of their fragments, according to some chemical reaction rules. We are now revising SmartGrow into igrow, utilizing idock as the backend docking engine, supporting more reaction rules, and tracking synthetic feasibility. We propose to integrate igrow into idock 3.0 to realize data caching and partial docking. We will exploit click chemistry for realistic synthetic feasibility, and implement dual-ligand docking.

In addition to technical development of novel tools, we have also applied our tools to practical drug discovery problems in real life. In a case study, we are working on discovering inhibitors of viral proteins of influenza A virus H1N1. We have identified the nucleoprotein and the RNA polymerase subunits PA and PB2 as druggable protein targets, and spent six months in running idock to test over 10 million ligands. We have shortlisted a few promising ligands to carry out subsequent biological assays, and we are now seeking for appropriate vendors. In another case study, we are working on discovering inhibitors of CCRK (Cell Cycle-Related Kinase), which has been proved to involve in glioblastoma multiforme carcinogenesis, ovarian carcinomas and hepatocarcinogenesis. We have executed idock to test about 5 thousand approved drugs, and shortlisted 11 ligands for purchasing. In future case studies, we plan to work on cancer stem cells and lipid-lowering drugs.
