Drug discovery is an expensive and long-term business. It takes US\$1.8 billion over 13.5 years to develop a new drug. We therefore aim to develop a comprehensive computational framework for structure-based drug discovery, simulating the early phases of modern drug discovery pipeline in order to save money and time.

So far, we have developed idock for fast protein-ligand docking. It predicts binding conformations of ligands against given proteins and their binding affinities in terms of free energy. Compared with state-of-the-art AutoDock Vina, idock obtains a speedup of 6.3x to 10.4x, achieving a screening performance of 1.3 drug-like ligands per CPU minute.

Currently, we are developing istar as a SaaS platform for idock, and igrow for computational synthesis of potent ligands.
We plan to incorporate GPU acceleration into idock 2.0 with both CUDA and OpenCL.

In addition to technical development of novel tools, we have also utilized our tools to address practical drug discovery problems in real life. As case studies, we are working on influenza A virus H1N1, tumors and carcinomas, and cancer stem cells, identifying inhibitors of certain proteins of therapeutic interest.