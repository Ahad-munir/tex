Drug discovery is an expensive and long-term business. It takes US\$1.8 billion over 13.5 years to develop a new drug. Therefore, we aim to develop a computational framework for drug discovery with GPU acceleration, simulating the early phases of modern drug discovery process in order to save money and time.

Computationally speaking, the simulation typically involves structure-based virtual screening of ligands (i.e. small compounds), and computational synthesis of potent ligands. So far, we have developed two tools for this simulation purpose.

The first tool, idock, is used for fast structure-based virtual screening. It predicts both binding conformations of ligands against given proteins and their binding affinities. Compared with AutoDock Vina, idock obtains a speedup of 6.3x to 10.4x, achieving a screening performance of 1.3 drug-like ligands per CPU minute.

The second tool, igrow, is used for computational synthesis of potent ligands. It synthesizes ligands from scratch by incorporating fragments. Compared with AutoGrow, ligands generated by igrow retain 100 Da lower molecular weights and 10\% lower free energies on average. In addition, igrow runs 30\% faster.

We have used our three new tools together with some other existing tools to discover potential new drugs for the treatments of acquired immune deficiency syndrome (AIDS) and Alzheimer's disease (AD). Several promising ligands have been computationally identified and syntehsized for further clinical investigations.

The development of our three new tools for drug discovery is just the start. Ultimately we aim to develop a comprehensive and uniform framework incorporating capabilities of binding site identification, molecular docking, virtual screening, ligand synthesis, drug properties prediction, and interactive visualization. Eventually and most importantly, we should utilize this framework to discover new drugs in order to save millions of human lives.

