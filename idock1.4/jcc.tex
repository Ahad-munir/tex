% Modified for use with JCC - Madhusudan Singh Copyright (C) (2012). All rights reserved.
\documentclass[12pt]{article}

\setlength{\oddsidemargin}{0in}  %left margin position, reference is one inch
\setlength{\textwidth}{6.5in}    %width of text=8.5-1in-1in for margin
\setlength{\topmargin}{-0.5in}    %reference is at 1.5in, -.5in gives a start of about 1in from top
\setlength{\textheight}{9in}     %length of text=11in-1in-1in (top and bot. marg.) 
\newenvironment{wileykeywords}{\textsf{Keywords:}\hspace{\stretch{1}}}{\hspace{\stretch{1}}\rule{1ex}{1ex}}

\usepackage{amsmath,amssymb}
\usepackage{graphicx}% Include figure files
%\usepackage{caption}
\usepackage{color}% Include colors for document elements
\usepackage{dcolumn}% Align table columns on decimal point
\usepackage{bm}% bold math
\usepackage[numbers,super,comma,sort&compress]{natbib}
%\usepackage[nolists, nomarkers, figuresfirst]{endfloat}

\definecolor{background-color}{gray}{0.98}

\title{A fundamental revolution in computational chemistry: final explanation for hitherto unexplained measurements of property X and predictions of property Y}
\author{Author A\thanks{Department of Biology, University 1, ...}, Author B\thanks{Department of Chemistry, University 2, ...}, Author C \thanks{Department of Physics, University 3, ...} Author D\thanks{Department of Mathematics, University 2, ...}}

\begin{document}

\maketitle


\begin{abstract}
Lorem ipsum dolor sit amet, consectetur adipisicing elit, sed do eiusmod tempor incididunt ut labore et dolore magna aliqua. Ut enim ad minim veniam, quis nostrud exercitation ullamco laboris nisi ut aliquip ex ea commodo consequat. Duis aute irure dolor in reprehenderit in voluptate velit esse cillum dolore eu fugiat nulla pariatur. Excepteur sint occaecat cupidatat non proident, sunt in culpa qui officia deserunt mollit anim id est laborum.
%((Insert abstract here))
\end{abstract}

\begin{wileykeywords}
One, Two, Three, Four, Five.
%A list of five key words or phrases which best characterize the paper are required for indexing.
\end{wileykeywords}

\clearpage

%*****************Graphical Table of Contents******************** THIS IS MANDATORY *******************


\begin{figure}[h]
\centering
\colorbox{background-color}{
\fbox{
\begin{minipage}{1.0\textwidth}
\includegraphics[width=55mm,height=50mm]{cc.eps} % Pick only one of the two styles by uncommenting the corresponding \includegraphics
%\includegraphics[width=110mm,height=20mm]{cc.eps}
\\
(75 words.) Images for the graphical Table of Contents should capture the essence of a paper, displaying a figure, plot, or scheme that is central to the theme of the manuscript. The text of the graphical Table of Contents is meant for the non-specialist and should ideally contain no obscure jargon or mathematical symbols / equations, but should attempt to convey the gist of the paper in everyday terms, while remaining consistent with accepted standards of scientific literature.
\end{minipage}
}}
\end{figure}

% makes references listed with 1., 2., etc.  
  \makeatletter
  \renewcommand\@biblabel[1]{#1.}
  \makeatother

\bibliographystyle{apsrev}

\renewcommand{\baselinestretch}{1.5}
\normalsize


\clearpage


\section*{\sffamily \Large SUMMARY}

%A short summary of the main contributions in the paper is required. This summary should be carefully prepared for it is automatically the source for most abstracts.

\section*{\sffamily \Large INTRODUCTION} % Not needed for rapid communications

%((Place Introduction here))

%((Main text paragraphs should be 12 point font, double-spaced. Reviews are comprehensive survey of recent progress in a topic of broad interest in quantum chemistry, providing the readership with an appreciation of the importance of the work, a summary of recent developments, and a guide to the relevant literature. Perspective are short discussions of an important emerging topics in quantum chemistry, usually focused on no more than a few recently published papers, and including the authors' vision for the future of the topics, identifying important problems that should be addressed next. Perspective should be limited to 3000 words, 4 display items (figures and/or tables), and 30 references.))

%((Full Papers are comprehensive reports of important recent advances in the development of basic theory, quantum mechanical computational methodologies and their relevant applications that provide significant insight to problems of broad interest in chemistry, physics, biology, and materials science. The opening sentence of the manuscript should summarize the reasons for the undertaking of the work and the main conclusions that can be drawn. The main text should be contain sections with brief subheadings, a summary of the major conclusions of the paper, and a Method section containing sufficient detail to reproduce the work. Main text paragraphs should be 12 point font, double-spaced.))

%((Rapid communications should be limited to 1500 words, 3 display items (figure and/or tables), 20 references. Main text paragraphs should be 12 point font, double-spaced. There are NO headings in the main text.))

\section*{\sffamily \Large METHODOLOGY}

%((Place Computational Methods here. Not needed for review articles))

%((Computational results should be prepared following the IUPAC guidelines (See Journal of Computational Chemistry, 20: 1587-1590 and 20:1591-1592). In particular, it is required that the level of theory employed is appropriate to the problem at hand, and that the sufficient details about methodology are provided to allow the work to be reproduced.)

%((In full papers, this section appears immediately after the introduction. In Rapid Communications, this section appears just before the Acknowledgments.))

\section*{\sffamily \Large RESULTS}

%((Place Results here. Not needed for review articles.))

\section*{\sffamily \Large First-order heading}
 
%((Equations should be inserted using standard LaTeX equation and eqnarray environments, not as graphics, and should be set in the main text))
%Equation											(1)
%((References should be superscripted and appear after punctuation.1,2 Please define all acronyms at their first usage except IR, UV, NMR, and DNA or similar commonly understood terms.)) 


\subsection*{\sffamily \large Second-order heading}

\subsubsection*{\sffamily \normalsize Third-order heading}

{\sffamily \small Fourth-order heading}\\

\section*{\sffamily \Large DISCUSSION}

%((Place Discussion here. Not needed for review articles.))


\section*{\sffamily \Large CONCLUSIONS}

%((Place Conclusions here.))

\subsection*{\sffamily \large ACKNOWLEDGMENTS}

%((Place Acknowledgments here))


%((Additional Supporting Information may be found in the online version of this article.))

\clearpage

%%%%%%%%%%%%%%%%%%%%%%%%%%%%%%%%%%%%%%%%%%%%%%%%%%%%%%%%%%%%%%%%%%%%%%%%%%%%%%%%%
% BIBLIOGRAPHY

%\bibliography{bibtexrefs}   % Produces the bibliography via BibTeX.

\begin{thebibliography}{99}


\bibitem{Coulson}
Coulson, C. A., Rev. Mod. Phys., \textbf{1960}, 32,170-177.
\bibitem{Malrieu}
Malrieu, J.-P., J. Mol. Struct., \textbf{1998}, 424, 1-2,83-91.
\bibitem{Shaik}
Shaik, S., New. J. Chem., \textbf{2007}, 31,2015-2028.
\bibitem{Hoffmann}
Hoffman, R., Schleyer, P. v. R., Schaefer III, H. F., \textbf{2008}, 47, 7164-7167.
\bibitem{Perdew}
Perdew, J. P., Ruzsinszky, A., Constantin, L., Sun, J., Csonka, G., J. Chem. Theory Comput., \textbf{2009}, 5, 902-908.
\bibitem{Koros}
Koros, W. J.; Chern, R. T. In Handbook of Separation Process Technology; Rousseau, E. D.; Russell, B., Eds.; Wiley: New York, \textbf{1987}; Vol. 2, Chapter 20, pp 34-45.
\end{thebibliography}


%%%%%%%%%%%%%%%%%%%%%%%%%%%%%%%%%%%%%%%%%%%%%%%%%%%%%%%%%%%%%%%%%%%%%%%%%%%%%%%%%

\clearpage
%%%%%%%%%%%%%%%%%%%%%%%%%%%%%%%%%%%%%%%%%%%%%%%%%%%%%%%%%%%%%%%%%%%%%%%%%%%%%%%%%
% FIGURE CAPTIONS

%%%%% FIGURE ---- cc.eps
\begin{figure}
\caption{\label{cc} Place Figure 1 caption here. In the case of reproduced figures in review articles, you must obtain the publisher's permission and state a suitable notice here along with a citation.}
\end{figure}

\begin{figure}
\caption{\label{fig2} Place Figure 2 caption here. Figures should be uploaded as individual files, preferably .tif or .eps files, at high enough resolution (600 to 1200 dpi) to ensure clarity. Please see the author’s guide for more details and specifications. For high quality illustrations, we highly recommend the use of the TikZ package.}
\end{figure}


%%%%%%%%%%%%%%%%%%%%%%%%%%%%



%%%%%%%%%%%%%%%%%%%%%%%%%%%%%%%%%%%%%%%%%%%%%%%%%%%%%%%%%%%%%%%%%%%%%%%%%%%%%%%%%
% FIGURE FILES

\clearpage

%\vspace*{0.1in}   %%% FIGURE 1
\begin{center}
\includegraphics[width=0.2\columnwidth,keepaspectratio=true]{cc.eps}
\end{center}
\vspace{0.25in}
\hspace*{3in}
{\Large
\begin{minipage}[t]{3in}
\baselineskip = .5\baselineskip
Figure 1 \\
Author A, Author B, Author C, Author D \\
J.\ Comput.\ Chem.
\end{minipage}
}

\clearpage

\begin{table}
\begin{tabular}{|c|c|c|c|}\hline
\textbf{Quantity} & \textbf{Calculated} & \textbf{Observed} & \textbf{Error} \\ \hline
  Density & 5.3 & 6.3 & Within limits \\ \hline
  Optical magnification & 8.3 & 90.9 & Utterly unacceptable\! \\ \hline
\end{tabular}
\caption{\label{tbl1} Place table caption here.}
\end{table}

\end{document}

